%!TEX root = ../tommaso-thesis.tex

\chapter{Pharo}\label{ch:pharo}
why do we use pharo



We collected large volumes of data from development sessions of users of \textit{Pharo}~\cite{black2009}, an object-oriented programming language  and companion IDE inspired by \textit{Smalltalk}.



An interesting property of Pharo comes from its dynamic nature: the whole system is polymorphic, and polymorphism is obtained through the so-called \emph{duck typing}~\cite{Chugh2012}: every object can be used in place of other objects, as long as it is able to respond to the same messages.
This entails that---as in other dynamic programming languages---there is no static type system and, as such, no static type checking: every type error happens at runtime, resulting in a \emph{Message Not Understood} kind of exception.
This peculiarity is important when considering the nature of exceptions in Pharo, because the vast majority of the exceptions is caused in this context: In our dataset an exception is thrown as a result of a message not understood in more than 72\% of the cases.
Among those cases, 68\% are generated from a message sent to \emph{UndefinedObject}.
These are the equivalent of a \emph{NullPointerException} in Java.



We chose to implement and test the effective feasibility of our approach using \emph{Pharo}\footnote{https://pharo.org}, an Object Oriented programming language inspired by Smalltalk.
Pharo inherits a number of powerful properties from its Smalltalk origins, that can support our task of implementing the data collection framework.
In particular, it is a \emph{live} programming environment, with full reflectivity capabilities, and a control over the whole system that allows to access and manipulate programmatically the complete state of the program.
Using Pharo in our context provides several abstractions over the entities of the system, that result in a technical advantage that alleviates us from fighting with several minor side effects.




\subsection{The Pharo Ecosystem}%\label{sec:blend-ingredients:pharo}

\pha is a Smalltalk inspired programming environment, composed of the \pha programming language, an integrated development editor and a set of libraries covering the common needs for the daily programming tasks.

Apart from a rich software collection, the \pha ecosystem is composed of a vibrant and active community\seeurl{http://pharo.org/community} that includes about 2,000 developers both from academia and industry.
The community actively participates in the development of the system by building tools to improve the user experience, submitting bug reports and proposing patches to solve defects.
