%!TEX root = ../tommaso-thesis.tex
%!TEX spellcheck = en_US

\begin{abstract}
Software development has become a prominent activity, with influence in almost any human activity.
To be able to fit in so many different scenarios and constantly implement new features, software developers adopted methodologies with tighter development cycles, with up to several releases per day.
%Software development is an activity centered around the continuous modification of source code to implement new features, improve the current behavior, and correct defects.
With the constant growth of modern software projects and the consequent expansion of development teams, understanding all the components of a system becomes an impossible task.

In this context understanding the cause of an error or identifying its source is not an easy task, and correcting the erroneous behavior can lead to unexpected unavailability of crucial services.
Being able to keep track of software defects, often referred to as \emph{bugs}, is crucial in maintaining a project while reducing maintenance costs.
For this purpose, the correctness and completeness of the information available has a great impact on the time required to understand an solve a problem.

In this thesis we present an overview of the current techniques commonly used to report software defects.
We show why we believe that the state of the art needs to be improved, and what approach we suggest in dealing with this data.
We then present a set of tools and approaches to collect data from software failures, model it and extract knowledge from it.
Our goal is to show that data generated from errors can have a great impact on modern software development, and how it can be employed to augment the development environment to assist software engineers to build and maintain software ecosystems.

\end{abstract}